\documentclass{jsarticle}
\usepackage{listings,jlisting}
\usepackage[dvipdfmx]{graphicx}

\lstset{%
	language={Java},
	basicstyle={\small},%
	identifierstyle={\small},%
	commentstyle={\small\itshape},%
	keywordstyle={\small\bfseries},%
	ndkeywordstyle={\small},%
	stringstyle={\small\ttfamily},
	frame={tb},
	breaklines=true,
	columns=[l]{fullflexible},%
	numbers=left,%
	xrightmargin=0zw,%
	xleftmargin=3zw,%
	numberstyle={\scriptsize},%
	stepnumber=1,
	numbersep=1zw,%
	lineskip=-0.5ex%
}


\begin{document}

\title{\vspace{-5cm}SDNを用いたIoTデバイスの相互通信の簡略化\vspace{-1em}}
\author{\vspace{-1em}}
\maketitle
\vspace{-1em}
\section{IoTとは}
Internet of Thingsの略で、物のインターネットとも呼ばれる。
物理現象を計測するセンサ,計測結果を解析するロジック,それによって,何かをもたらすアクチュエータがそれぞれインターネットに接続し,相互通信することにより,様々な自動化が図られている。
様々なことへの応用が考えられており,期待されている分野である.
\section{IoTの事例}
\begin{itemize}
	\item 
	\item 
	\item
\end{itemize}
\section{IoTの課題}

\section{関連するもの}
\end{document}

